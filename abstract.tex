% abstract.tex
%% Create a abstract page, as specified by the Course-VI M.Eng. Thesis Guide.
%% Careful to use the special "abstractpage" environment here, rather than the
%% usual "abstract" environment.
\begin{abstractpage}
\pdfbookmark[0]{Abstract}{abstract} % Sets a PDF bookmark for the abstract

The proliferation of dynamic program analysis tools has done much to ease the
burden of developing complex software.  However, creating such tools remains a
challenge.  Dynamic binary instrumentation frameworks such as DyanamoRIO and
Pin provide support for such tools by taking responsibility for application
transparency and machine code manipulation.  However, tool writers must still
make a tough choice when writing instrumentation: should they inject custom
inline assembly into the application code, or should they use the framework
facilities for inserting callbacks into regular C code?  Custom assembly can be
more performant and more flexible, but it forces the tool to take some
responsibility for maintaining application transparency.  Callbacks into C, or
``clean calls,'' allow the tool writer to ignore the details of maintaining
transparency.  Generally speaking, a clean call entails switching to a safe
stack, saving all registers, materializing the arguments, and jumping to the
callback.

% Too much exposition above.

This thesis presents a suite of optimizations for DynamoRIO that improve the
performance of ``na\"ive tools,'' or tools which rely primarily on clean calls
for instrumentation.  The contributions include:

\begin{itemize}
\item Inlining simple, leaf callbacks into the application code stream.
\item Partial inlining of callbacks which perform a simple check to decide
between doing work which can be inlined and work that cannot.
%\item Lean call insertion, which inserts a call with minimal code footprint.
\item Coalescing inlined calls to prevent unnecessary register saves or stack
switches.
%\item Transforming instructions to avoid usage of the x86 flags register.
%\item Using fewer registers to eliminate extra saves and restores.
\item Classic optimizations such as dead store elimination and redundant load
elimination.
\end{itemize}

With this additional functionality built on DynamoRIO, we have shown
improvements of up to 54.8x for a na\"ive instruction counting tool as well as a
3.7x performance improvement for a memory alignment checking tool on average for
many of the benchmarks from the SPEC 2006 CPU benchmark suite.

\end{abstractpage}
