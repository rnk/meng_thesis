\chapter{Inlining}
\label{sec:inlining}

\section{Instruction Count Tool}

The first tool we examine in this thesis is instruction count.  The goal of this
tool is to compute the exact number of instructions that would be executed by
the application if we were to run it natively.  The simplest way of achieving
this goal is to iterate across each instruction and insert a call to a routine
which increments a 64-bit counter before executing the routine.  A fragment of
the source code for this tool is shown in Figure \ref{fig:naive_inscount}.

\begin{figure}
\begin{verbatim}
#include "dr_api.h"
#include "dr_calls.h"

/* ... */

static void inc_count(void) {
    global_count++;
}

dr_emit_flags_t
event_basic_block(void *drcontext, void *tag, instrlist_t *bb,
                  bool for_trace, bool translating) {
    instr_t *instr;
    for (instr = instrlist_first(bb); instr != NULL;
         instr = instr_get_next(instr))
        drcalls_insert_call(drcontext, bb, instr, (void *)inc_count,
                            false /* save fpstate */, 0);
    /* ... */
}
\end{verbatim}
\caption{Abbreviated source code sample for our instruction count tool.}
\label{fig:naive_inscount}
\end{figure}

Instruction count is a simple example, and it would be fair to suggest that in
this case a reasonable tool author would be able to optimize this to a single
instrumentation routine call in each basic block.  However, we believe that for
more complicated tools, this is not easy, and inserting many calls to
instrumentation routines around the instructions they pertain to is the simple
and expected way to write the tool.  For example, a tool like {\tt opcodemix},
a sample Pin tool that reports the distribution of x86 instruction opcodes
executed, is implemented the same way.  The simple way to implement {\tt
opcodemix} is to instrument each instruction with a call to a function that
takes the instruction opcode as a parameter.  The routine then increments the
appropriate counter based on the opcode.  We expect that all optimizations
applying to our sample instruction count tool would be just as effective for
{\tt opcodemix} as well as other similar tools.

\section{Clean Calls}
\label{sec:clean_calls}

Implemented with the standard clean call infrastructure that DynamoRIO provides,
this simple tool creates large fragments of code that perform poorly.  Figure
\ref{fig:clean_call} shows an x86\_64 assembly listing of a clean call.

\begin{figure}
\begin{verbatim}
mov    %rsp -> %gs:0x00                                 # Switch to dstack
mov    %gs:0x20 -> %rsp
mov    0x000002c0(%rsp) -> %rsp
lea    0xfffffd50(%rsp) -> %rsp                         # Allocate frame
mov    %rax -> 0x48(%rsp)                               # Save GPRs
... save rbx-r14 ...
mov    %r15 -> 0x00000088(%rsp)
lahf   -> %ah
seto   -> %al
mov    %rax -> 0x00000090(%rsp)                         # Save "aflags"
movdqa %xmm0 -> 0x000000b0(%rsp)                        # Save XMM regs
... save xmm1-xmm14 ...
movdqa %xmm15 -> 0x00000290(%rsp)
call   $0x004041c0 %rsp -> %rsp 0xfffffff8(%rsp)        # Call inc_count
movdqa 0x000000b0(%rsp) -> %xmm0                        # Restore XMM regs
... restore xmm1-xmm14 ...
movdqa 0x00000290(%rsp) -> %xmm15
mov    0x00000090(%rsp) -> %rax                         # Restore "aflags"
add    $0x7f %al -> %al
sahf   %ah
mov    0x48(%rsp) -> %rax                               # Restore GPRs
... restore rbx-r14 ...
mov    0x00000088(%rsp) -> %r15
mov    %gs:0x00 -> %rsp                                 # Restore appstack
\end{verbatim}
\caption{Assembly listing for a single clean call.}
\label{fig:clean_call}
\end{figure}

Although all we want to do is increment a counter, we end up doing all of the
following work in order to support as much arbitrary C code as possible:

\begin{packed_enumerate}
\item Switch to a clean stack.
\item Save all registers.
\item Save the flags register.
\item Save ``volatile'' XMM registers.
\item Materialize arguments into parameter registers.  Not shown above, since
inc\_count takes no arguments.
\item Call tool routine.
\item Restore ``volatile'' XMM registers.
\item Restore the flags register.
\item Restore all registers.
\item Switch back to the application stack.
\end{packed_enumerate}

Saving the XMM registers seems excessive for a short snippet of C code, but even
if the tool is avoiding the use of floating point instructions, it is still
possible for the compiler to choose to use XMM registers.  More likely, however,
is that the tool may call {\tt memcpy}.  In recent versions of glibc, {\tt
memcpy} will make use of the XMM registers to widen the load and store width,
which will clobber the caller-saved XMM registers.  Therefore DynamoRIO must
conservatively save this set of registers, even though they are most likely
unused.

For the SPEC bzip2 benchmark, we get terrible performance.  Natively, this
application runs in 4.5 seconds.  With clean calls, the application takes 9
minutes and 20 seconds, giving a 124x slowdown.  Through the rest of the
chapter we show how to improve on that.

Our primary target is to avoid doing all this extra work of context switching.
Our first technique is to {\em inline} the instrumentation routine into the
application code stream.

\section{Simple Inlining}
\label{sec:simple_inlining}

For inlining routines that have no control flow, the process is simple.  The
tool provides us with a function pointer which is the PC of the first
instruction in the routine.  We start decoding instructions from this PC until
we find the first {\tt ret} instruction.  Assuming the routine has no control
flow, this is adequate for reliably decoding the routine.  When we extended our
inliner to perform partial inlining we had to augment our decoding algorithm as
described in Section \ref{sec:decoding_cti}.

Inlining is only possible if we are able to analyze the callee and only if the
analysis suggests that doing so would be profitable.  The analysis is designed
so that if there are any corner cases that the analysis cannot handle reliably,
it can fail, and inlining will fail.  For example, if the stack pointer is
dynamically updated in the middle of the function, this inhibits stack usage
analysis.  Therefore we bail out in such cases.  Assuming our analysis is
successful, we only inline if the following criteria apply:

\begin{packed_itemize}
\item The callee is a leaf function.  Our definition of leaf function is
conservative, meaning it must not have any calls, trapping instructions,
indirect branches, or direct branches outside of the function.
\item The simplified callee instruction stream is no longer than 20
instructions.  This limit is in place to avoid code bloat.  The limit of 20
instructions has not been experimentally tested and was chosen to match Pin's
behavior.
\item The callee does not use XMM registers.
\item The callee does not use more than a fixed number of general purpose
registers.  We have not picked an appropriate limit yet.
\item The callee must have a simple stack frame that uses at most one stack
location.
\item The callee may only have as many arguments as can be passed in registers
on the current platform, or only one if the native calling convention does not
support register parameters.
\end{packed_itemize}

If the routine meets the above criteria, we can then simplify the clean call
process described in Section \ref{sec:clean_calls} down to the following steps:

\begin{packed_enumerate}
\item Switch to a clean stack.
\item Save clobbered registers.
\item Save the flags register if used.
\item Materialize arguments into parameter registers.
\item Emit the simplified inline code stream.
\item Restore the flags register if used.
\item Restore clobbered registers.
\item Switch back to the application stack.
\end{packed_enumerate}

In particular, we skip all XMM saves and most of the register saves, especially
on x86\_64 which has more registers than most callees use.

For our instruction count tool, the disassmbled {\tt inc\_count} routine is
shown in its original form in Figure \ref{fig:inc_count_asm} and after inlining
in Figure \ref{fig:inscount_inlined}.  Note how all the stack frame setup and
teardown related instructions from the original routine have been removed in the
inlined version, and been replaced with our own stack management.

\begin{figure}
\begin{verbatim}
push   %rbp %rsp -> %rsp 0xfffffff8(%rsp)       # Frame setup
mov    %rsp -> %rbp 
mov    <rel> 0x00000000006189e0 -> %rax         # Load global_count
add    $0x0000000000000001 %rax -> %rax         # Increment
mov    %rax -> <rel> 0x00000000006189e0         # Store global_count
leave  %rbp %rsp (%rbp) -> %rsp %rbp            # Frame teardown
ret    %rsp (%rsp) -> %rsp                      # ret
\end{verbatim}
\caption{Assembly for the {\tt inc\_count} instrumentation routine.}
\label{fig:inc_count_asm}
\end{figure}

\begin{figure}
\begin{verbatim}
mov    %rsp -> %gs:0x00                         # Switch to dstack
mov    %gs:0x20 -> %rsp 
mov    0x000002c0(%rsp) -> %rsp 
lea    0xfffffd50(%rsp) -> %rsp                 # Create frame
mov    %rax -> 0x48(%rsp)                       # Save %rax, only used reg
lahf   -> %ah                                   # Save aflags
seto   -> %al 
mov    %rax -> 0x00000090(%rsp) 
mov    <rel> 0x00000000006189e0 -> %rax         # Inlined increment code
add    $0x0000000000000001 %rax -> %rax 
mov    %rax -> <rel> 0x00000000006189e0 
mov    0x00000090(%rsp) -> %rax                 # Restore aflags
add    $0x7f %al -> %al 
sahf   %ah 
mov    0x48(%rsp) -> %rax                       # Restore %rax
mov    %gs:0x00 -> %rsp                         # Back to appstack
mov    (%rbx) -> %rax                           # APPLICATION INSTRUCTION
... Process repeats for each application instruction
\end{verbatim}
\caption{{\tt inc\_count} inlined into an application basic block without
further optimization.}
\label{fig:inscount_inlined}
\end{figure}

The inlined version of the code is clearly an improvement over the clean call
version of the code, and is 5.8x faster than the clean call version, running in
96.3 seconds.  Still the cost of switching contexts to the DynamoRIO stack to
save registers dwarfs the cost of the actual code we wish to execute.  While
there are opportunities for reducing the work done here, we introduce the novel
technique of ``call coalescing'' to reduce the number of switches needed.

\section{Call Coalescing}

In order to do coalescing, we need visibility of the entire instrumented basic
block.  Therefore we choose to represent calls to instrumentation routines as
artifical pseudo-instructions that are expanded after instrumentation.  Before
expanding each call, we see if it is possible to group them together within the
basic block.  If the call has register arguments, we avoid moving it outside
the live range of its register arguments or past any memory writes.  This
approach works well for instruction count and other tools that pass immediate
values to instrumentation routines, because we have total freedom as to how
they are scheduled.

Once the calls are scheduled, we expand them to another level of pseudo-ops of
stack switches and register saves and register restores.  Before lowering these
pseudo-ops, we pass over the instruction stream to delete consecutive
reciprocal operations.  Currently we identify three pairs of operations as
being reciprocal: switching to dstack and back, restoring and saving a
register, and restoring and saving flags.  Coalescing all calls in a three
instruction basic block results in the code in Figure
\ref{fig:inscount_coalesced}.

\begin{figure}
\begin{verbatim}
mov    %rsp -> %gs:0x00                         # Switch to dstack
mov    %gs:0x20 -> %rsp 
mov    0x000002c0(%rsp) -> %rsp 
lea    0xfffffd50(%rsp) -> %rsp                 # Setup frame
mov    %rax -> 0x48(%rsp)                       # Save %rax
lahf   -> %ah                                   # Save flags
seto   -> %al 
mov    %rax -> 0x00000090(%rsp) 
mov    <rel> 0x0000000000618c80 -> %rax         # Call 1
add    $0x0000000000000001 %rax -> %rax 
mov    %rax -> <rel> 0x0000000000618c80 
mov    <rel> 0x0000000000618c80 -> %rax         # Call 2
add    $0x0000000000000001 %rax -> %rax 
mov    %rax -> <rel> 0x0000000000618c80 
mov    <rel> 0x0000000000618c80 -> %rax         # Call 3
add    $0x0000000000000001 %rax -> %rax 
mov    %rax -> <rel> 0x0000000000618c80 
mov    0x00000090(%rsp) -> %rax                 # Restore flags
add    $0x7f %al -> %al 
sahf   %ah 
mov    0x48(%rsp) -> %rax                       # Restore %rax
mov    %gs:0x00 -> %rsp                         # Restore appstack
... three application instructions
\end{verbatim}
\caption{Three inlined calls to {\tt inc\_count} coalesced together.}
\label{fig:inscount_coalesced}
\end{figure}

By coalescing calls together we achieve a 2.1x improvement over standard
inlining, bringing our execution time down to 45.6 seconds.  However, we still
spend an entire four instructions setting up a stack frame just so that we can
save {\tt \%rax} and the flags register.  In the next section, we describe how
we can avoid switching stacks altogether.

\section{Avoiding Stack Switching with TLS}
\label{sec:tls_scratch}

The original reason for switching stacks is that we cannot rely on being able to
use the stack used by the application.  There are many reasons why this might be
the case.  The application might be using the red zone on Linux x86\_64, which
is an area past the end of the stack usable by leaf functions.  The application
might be close to the guard page which will grow the stack.  The application
might have a dangling pointer to the stack and we may want to find that bug.
Finally, the application might be running without a stack and just using {\tt
\%rsp} as a general purpose register.  In any case, it is important to assume
nothing about the application stack.

Our main use for the stack when inlining code is as temporary scratch space
where we can save registers that are clobbered.  For single-threaded
applications we can simply use global variables, but this breaks down quickly if
we want to share the code cache between threads.  In order to have per-thread
global variables, we need to use ``thread local storage,'' or TLS.  On Windows
and Linux the x86 {\tt fs} and {\tt gs} segment registers are used to point to a
memory region unique for each thread.  On Linux, only one segment is used, so we
claim the other and use it for our own TLS memory region.  On Windows, however,
the operating system will not preserve our segment register value across
interrupts, so we cannot set our own.  Instead, we steal TLS space from the
application by introspecting on the application's thread execution block (TEB)
and marking our storage as allocated.\cite{inside_win2k}  DynamoRIO uses a small
amount of this space internally, and we cannot afford to allocate too much or we
will interfere with the execution of the application.  Therefore we do not
enable this optimization by default.  Finally, if we wish to perform partial
inlining as described in Chapter \ref{sec:partial_inlining}, we cannot yet
transfer from the inline code to the slowpath without an existing stack frame.

Figure \ref{fig:inscount_tls} shows the instruction count example code from the
previous section with the TLS scratch space optimization turned on.  Note that
it does not need any stack frame setup, and simply saves registers and flags
into TLS slots.

\begin{figure}
\begin{verbatim}
mov    %rax -> %gs:0x00000088                   # Save %rax
lahf   -> %ah                                   # Save flags
seto   -> %al 
mov    %rax -> %gs:0x00000090 
mov    <rel> 0x0000000000619780 -> %rax         # Call 1
add    $0x0000000000000001 %rax -> %rax 
mov    %rax -> <rel> 0x0000000000619780 
mov    <rel> 0x0000000000619780 -> %rax         # Call 2
add    $0x0000000000000001 %rax -> %rax 
mov    %rax -> <rel> 0x0000000000619780 
mov    <rel> 0x0000000000619780 -> %rax         # Call 3
add    $0x0000000000000001 %rax -> %rax 
mov    %rax -> <rel> 0x0000000000619780 
mov    %gs:0x00000090 -> %rax                   # Restore flags
add    $0x7f %al -> %al 
sahf   %ah 
mov    %gs:0x00000088 -> %rax                   # Restore %rax
... three application instructions
\end{verbatim}
\caption{Three inlined and coalesced calls to {\tt inc\_count} that use TLS
instead of dstack.}
\label{fig:inscount_tls}
\end{figure}

Using TLS space on this example only gains us about a 1\% performance
improvement.  When we initially implemented this optimization, we obtained much
better results, because we implemented TLS usage before call coalescing.  After
coalescing, there are fewer stack switches to worry about, so we are only able
to delete 3 frame setup instructions.

To address the next source of inefficiency in our instruction counting tool, we
use some classic compiler optimizations on the inlined code, now that the
register spilling has been sufficiently optimized.

\section{Redundant Load Elimination and Dead Store Elimination}

Redundant load elimination (RLE) and dead store elimination (DSE) are classic
compiler optimizations that we found useful for optimizing tools.  In our last
example, the inlined code is loading {\tt global\_count}, incrementing it, and
storing it three separate times.  Rather than having three separate loads and
stores, we would like to have one load at the beginning and one store at the
end.

The logic for our version of RLE is that if there is a store to and a load from
the same memory operand without any potentially aliasing memory writes between
them, we can use the source register from the store instead of loading the value
again.  We have to be careful that the register that was written to the memory
location is still live by the time the load occurs, and that it is live for all
uses of the load's destination register.  If that is the case, then we rewrite
all uses of the load's destination register over its live range to use the
store's source register.  Figure \ref{fig:inscount_rle} shows the example code
fragment after applying RLE.

\begin{figure}
\begin{verbatim}
mov    %rax -> %gs:0x00000088                   # Save %rax
lahf   -> %ah                                   # Save flags
seto   -> %al 
mov    %rax -> %gs:0x00000090 
mov    <rel> 0x0000000000619780 -> %rax         # Load 1
add    $0x0000000000000001 %rax -> %rax 
mov    %rax -> <rel> 0x0000000000619780         # Store 1
add    $0x0000000000000001 %rax -> %rax         # Load 2 is gone
mov    %rax -> <rel> 0x0000000000619780         # Store 2
add    $0x0000000000000001 %rax -> %rax         # Load 3 is gone
mov    %rax -> <rel> 0x0000000000619780         # Store 3
mov    %gs:0x00000090 -> %rax                   # Restore flags
add    $0x7f %al -> %al 
sahf   %ah 
mov    %gs:0x00000088 -> %rax                   # Restore %rax
... three application instructions
\end{verbatim}
\caption{Three inlined calls to {\tt inc\_count} after applying just RLE.}
\label{fig:inscount_rle}
\end{figure}

Dead store elimination (DSE) is similar, in that if there is a store to memory,
followed by no potentially aliasing reads from memory, followed by a store to
the same memory location, we can delete the first store.  Because we do not need
to rewrite any uses of any registers, this is simpler than RLE.  Figure
\ref{fig:inscount_rle_dse} shows the application of both RLE and DSE.

\begin{figure}
\begin{verbatim}
mov    %rax -> %gs:0x00000088                   # Save %rax
lahf   -> %ah                                   # Save flags
seto   -> %al 
mov    %rax -> %gs:0x00000090 
mov    <rel> 0x0000000000619780 -> %rax         # Load 1
add    $0x0000000000000001 %rax -> %rax         # Increment 1
add    $0x0000000000000001 %rax -> %rax         # Increment 2
add    $0x0000000000000001 %rax -> %rax         # Increment 3
mov    %rax -> <rel> 0x0000000000619780         # Store 3
mov    %gs:0x00000090 -> %rax                   # Restore flags
add    $0x7f %al -> %al 
sahf   %ah 
mov    %gs:0x00000088 -> %rax                   # Restore %rax
... three application instructions
\end{verbatim}
\caption{Three inlined calls to {\tt inc\_count} after applying RLE and DSE.}
\label{fig:inscount_rle_dse}
\end{figure}

After applying both optimizations, we get a satisfactory 2.1x speedup, bringing
the bzip2 execution time much closer to the native execution time at 21.1
seconds.  Still, our code is not satisfactory.  We are using the {\tt add}
instruction, which updates the carry, overflow, and other bits in the x86 flags
register.  We cannot assume that the application is not using these flag bits,
so we must preserve them, requiring an extra stack slot and some extra flag
manipulation instructions.  The next section describes how to avoid this.

\section{Flags Avoidance}

Many x86 arithmetic instructions modify the x86 flags register.  This is
problematic, because it means we have to save and restore the bits of the flags
register that we touch.  This can be done either using the {\tt pushf} and {\tt
popf} instructions, or a sequence of {\tt lahf}, {\tt seto}, {\tt lahf}, and
{\tt add}.  The {\tt popd} instruction turns out to be fairly expensive, because
it sets many complex execution flag like the single-stepping trap flag for
debuggers.  Also, the {\tt lahf} sequence avoids stack usage, so it is
preferred.

However, we have the opportunity to transform inlined code to avoid flags usage
if possible.  Instead of using the {\tt add} instruction to perform additions
and subtractions, we can use the {\tt lea} instruction.  The {\tt lea}
instruction was intended to be a ``load effective address'' instruction for
dealing with things like the complex x86 segmented address space model.  For its
source, we provide an arbitrary x86 memory operand, for which it computes the
effective address and stores the result in a destination register.  However, we
can provide a memory operand which is performing simple additions instead of
address arithmetic.  Because {\tt lea} is not designed for arithmetic, it does
not modify flags.

We have a pass which can automatically perform this transformation.  However,
because {\tt lea} does not support memory operands as destinations, we have to
introduce a temporary register along with a load and store if the arithmetic
operation uses a memory operand as a destination.  Our pass attemps to pick a
register which has already been used by the inlined code so that we do not need
to introduce any extra saves and restores.

We show the results of applying this pass to our repeated additions in Figure
\ref{fig:inscount_flags}.  As shown in the listing, this is a big win, because
it eliminates three instructions on both sides of the call site as well as two
memory accesses.

\begin{figure}
\begin{verbatim}
mov    %rax -> %gs:0x00000088                   # Save %rax
mov    <rel> 0x0000000000619780 -> %rax         # Load global_count
lea    0x01(%rax) -> %rax                       # No flags usage
lea    0x01(%rax) -> %rax                       # No flags usage
lea    0x01(%rax) -> %rax                       # No flags usage
mov    %rax -> <rel> 0x0000000000619780         # Store global_count
mov    %gs:0x00000088 -> %rax 
... three application instructions
\end{verbatim}
\caption{Three inlined calls to {\tt inc\_count} after avoiding aflags usage.}
\label{fig:inscount_flags}
\end{figure}

After applying our flags avoidance optimization, we obtain a 1.4x speedup over
our previous time.  Still, it is obvious to a human that this is not as
efficient as it could be.  We do not need three separate increment instructions.
They can be folded together to a single {\tt lea} that adds 3 to {\tt \%rax}.

\section{Folding {\tt lea} Instructions}
\label{sec:fold_lea}

Using some of the same register liveness analysis tools we developed to support
our RLE and DSE optimizations, we can do the same register rewriting for {\tt
lea} operations.  If one {\tt lea} defines a register used in another {\tt lea},
we can attempt to combine the two memory operands.  For simple base and
displacement memory operands, as in the listing shown in the previous section,
this is as simple as adding the two displacements together and substituting in
the original base register.  We are also able to handle more complex cases
involving two memory operands both of which have a base, displacement, index,
and scale.  After we apply our instruction folding optimizations to the example
at hand, we get the assembly listing in Figure \ref{fig:inscount_lea}.

\begin{figure}
\begin{verbatim}
mov    %rax -> %gs:0x00000088                   # Save %rax
mov    <rel> 0x0000000000619780 -> %rax         # Load global_count
lea    0x03(%rax) -> %rax                       # Add 3
mov    %rax -> <rel> 0x0000000000619780         # Store global_count
mov    %gs:0x00000088 -> %rax                   # Restore %rax
... three application instructions
\end{verbatim}
\caption{Three inlined calls to {\tt inc\_count} after avoiding aflags usage.}
\label{fig:inscount_lea}
\end{figure}

Applying our {\tt lea} folding optimization gives us a 1.5x speedup over our
previous best, bringing our execution time down to 9.8 seconds.  The
application's native execution speed is 4.5 seconds on our hardware, and the
unoptimized instrumented execution time was 9 minutes and 20 seconds.  After
applying all of our optimizations we are able to improve the instrumented
execution time 57 fold, obtaining only a 2.1x slowdown from native execution.

\section{Chapter Summary}

Using the example of instruction counting, we have shown how our inliner and
suite of supporting optimizations can drastically improve the performance of
na\"ive instrumentation tools.  First, we started by looking at how a basic
clean call is implemented and at how it does more work than is necessary.  We
moved on to do basic inlining, where we switch stacks, save only the registers
and flags necessary to execute the routine inline, and execute it.  Using the
suite of optimizations developed during this thesis, we were able to
incrementally improve on the code generated in the first example.  For our first
optimization, using pseudo-instructions, we were able to schedule multiple calls
together and avoid duplicated work.  Next we avoided the stack switch altogether
by using thread local storage, which may not always be available.  Regardless,
we then moved on to look at how redundant load elimination and dead store
elimination can make a big difference in cleaning up our inlined code.  Finally,
we finished by looking at the low-level machine specific optimizations for
avoiding aflags usage and combining multiple arithmetic instructions.

In the next chapter, we look at two new examples of a memory alignment checker
and a memory trace tool too demonstrate the major contribution of this thesis:
partial inlining.
