\chapter{Contributions}
\label{sec:contributions}

Using the optimizations presented in this thesis, tool authors are able to
quickly build performant dynamic instrumentation tools without having to
generate custom machine code.

First, we have created an optimization which performs automatic inlining of
short instrumentation routines.  Our inlining optimization can achieve as much
as 50 times speedup as shown by our instruction count benchmarks.

Second, we have built a novel framework for performing partial inlining which
handles cases where simple inlining fails due to the complexity of handling
uncommon cases.  Partial inlining allows us to maintain the same callback
interface, while accelerating the common case of conditional analysis by almost
four fold.

Finally, we present a suite of standard compiler optimizations operating on
instrumented code.  With these optimizations, we are able to ameliorate the
register pressure created by inlining and avoid unecessary spills.  Without our
suite of optimizations, we would not be able to successfully inline many of our
example tools.

Once these tools have been contributed back to DynamoRIO, we hope to see more
tool authors use DynamoRIO to build quickly novel dynamic analysis tools that
run fast enough to be widely adopted.
