\chapter{Conclusion}

\section{Future Work}

While our techniques achieve great success with our instruction count example,
our parital inlining example performance falls short of our ambitions.  In order
to improve this, we need to consider a few things.

First, we should look into register re-allocation.  Pin uses a linear scan
register allocator to steal scratch registers from the application which it can
use in the inline code to avoid extra register spills.  If we used this
approach, we could eliminate most of the cost of a context switch.  Performing
register reallocation would be a large task requiring integration with the core
of DynamoRIO in order to accurately handle signal translation.

We should also consider providing general-purpose building blocks for common
tasks instead of asking tool authors to use plain C.  For example, we are
currently considering integrating some of the memory access collection routines
from DrMemory into a DynamoRIO extension.  With this support, tools would be
able to efficiently instrument all memory accesses without worrying about
whether their instrumentation meets our inlining criteria.

Another building block we could provide is general purpose buffer filling
similar to what was done in PiPa.\cite{pipa}  With a general purpose buffering
facility, tool authors do not need to worry about whether their calls were
inlined, and can be confident that DynamoRIO has inserted the most efficient
instrumentation possible.

Another improvement we could make is to allow the tool to expose the check to us
explicitly.  The idea is to have the tool give us two function pointers for
conditional analysis: the first returns a truth value indicating whether the
second should be called, and the second performs analysis when the first returns
true.  Pin uses this approach.  It requires the tool author to realize that this
API exists in order to take advantage of it, but it provides more control over
the inlining process.  We could provide a mechanism for requesting that a
routine be inlined regardless of criteria and raise an error on failure,
allowing the tool to know what routines were inlined.

\section{Contributions}
\label{sec:contributions}

Using the optimizations presented in this thesis, tool authors are able to
quickly build performant dynamic instrumentation tools without having to
generate custom machine code.

First, we have created an optimization which performs automatic inlining of
short instrumentation routines.  Our inlining optimization can achieve as much
as 50 times speedup as shown by our instruction count benchmarks.

Second, we have built a novel framework for performing partial inlining which
handles cases where simple inlining fails due to the complexity of handling
uncommon cases.  Partial inlining allows us to maintain the same callback
interface, while accelerating the common case of conditional analysis by almost
four fold.

Finally, we present a suite of standard compiler optimizations operating on
instrumented code.  With these optimizations, we are able to ameliorate the
register pressure created by inlining and avoid unnecessary spills.  Without our
suite of optimizations, we would not be able to successfully inline many of our
example tools.

Once these facilities have been contributed back to DynamoRIO, we hope to see
more tool authors choose DynamoRIO for its ease of use in building fast dynamic
instrumentation tools.
